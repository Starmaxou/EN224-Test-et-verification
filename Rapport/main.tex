\documentclass[a4paper]{article}

% Formatting
\usepackage[utf8]{inputenc}
\usepackage[margin=1in]{geometry}
\usepackage[titletoc,title]{appendix}

% Math
% https://www.overleaf.com/learn/latex/Mathematical_expressions
% https://en.wikibooks.org/wiki/LaTeX/Mathematics
\usepackage{amsmath,amsfonts,amssymb,mathtools}

% Images
% https://www.overleaf.com/learn/latex/Inserting_Images
% https://en.wikibooks.org/wiki/LaTeX/Floats,_Figures_and_Captions
\usepackage{graphicx,float,subfigure}

% Tables
% https://www.overleaf.com/learn/latex/Tables
% https://en.wikibooks.org/wiki/LaTeX/Tables

% Algorithms
% https://www.overleaf.com/learn/latex/algorithms
% https://en.wikibooks.org/wiki/LaTeX/Algorithms
\usepackage[ruled,vlined]{algorithm2e}
\usepackage{algorithmic}
\usepackage{listings}

% Code syntax highlighting
% https://www.overleaf.com/learn/latex/Code_Highlighting_with_minted
\usepackage{minted}
\usemintedstyle{borland}

% References
% https://www.overleaf.com/learn/latex/Bibliography_management_in_LaTeX
% https://en.wikibooks.org/wiki/LaTeX/Bibliography_Management
\usepackage{biblatex}
\addbibresource{references.bib}

\usepackage{xcolor}
\usepackage{listings}

\definecolor{mGreen}{rgb}{0,0.6,0}
\definecolor{mGray}{rgb}{0.5,0.5,0.5}
\definecolor{mPurple}{rgb}{0.58,0,0.82}
\definecolor{backgroundColour}{rgb}{0.95,0.95,0.92}

\lstdefinestyle{CStyle}{
    backgroundcolor=\color{backgroundColour},   
    commentstyle=\color{mGreen},
    keywordstyle=\color{magenta},
    numberstyle=\tiny\color{mGray},
    stringstyle=\color{mPurple},
    basicstyle=\footnotesize,
    breakatwhitespace=false,         
    breaklines=true,                 
    captionpos=b,                    
    keepspaces=true,                 
    numbers=left,                    
    numbersep=5pt,                  
    showspaces=false,                
    showstringspaces=false,
    showtabs=false,                  
    tabsize=2,
    language=C
}

\newcommand{\microAmp}{\mu A}

% Title content
\title{EN224 - Test et vérification}

\author{ALBERTY Maxime}
\date{9 février 2021}

\begin{document}

\maketitle

\tableofcontents

\newpage %-------Saut de Page---------

\section{Introduction}

\section{Software}
    \subsection{Etape 1}
        L'implémentation de la fonction PGCD fût une simple traduction de l'algorithme présenté en langage C.
        \begin{lstlisting}[style=CStyle]
        int PGCD(int A, int B)
        {
        	while(A != B){
        		if (A > B){
        			A = A - B;
        		} else {
        			B = B - A;
        		}
        	}
        	return A;
        }
        \end{lstlisting}
        
        Cependant il est important de resté concentré, car même si l'algorithme est trivial, une étourderie est vite arrivé et peut faire perdre du temps inutilement.
        \\
        
        Le test de la fonction est effectué au moyen de couples de valeurs d'on aura préalablement calculé le PGCD. 
        \begin{lstlisting}[style=CStyle]
        	printf("PGCD(1024,800) = %d\n",PGCD(1024,800));
        	printf("PGCD(800,1024) = %d\n",PGCD(800,1024));
        	printf("PGCD(32767,65535) = %d\n",PGCD(32767,65535));
        	printf("PGCD(65535,32767) = %d\n",PGCD(65535,32767));
        	printf("PGCD(512,2048) = %d\n",PGCD(512,2048));
        	printf("PGCD(2048,512) = %d\n",PGCD(2048,512));
        	printf("PGCD(458,6272) = %d\n",PGCD(458,6272));
        	printf("PGCD(6272,458) = %d\n",PGCD(6272,458));
        	printf("PGCD(783,125) = %d\n",PGCD(783,125));
        	printf("PGCD(125,783) = %d\n",PGCD(125,783));
        \end{lstlisting}
        Il faut ensuite comparé manuellement les résultats aux calculs.
        
        
    \subsection{Etape 2}
        Afin de pouvoir test plus de valeurs sans avoir à dupliquer les lignes de tests, nous ajoutons une génération aléatoires des valeurs de A et B comprisent entre 0 et 65535.
        \begin{lstlisting}[style=CStyle]
        \end{lstlisting}
        \begin{lstlisting}[style=CStyle]
        #define MAX_RAND 65535
        #define MIN_RAND 0 
        
        int RandA(void){
        	int A = (rand() % (MAX_RAND + 1 - MIN_RAND)) + MIN_RAND;
            return A;
        }
        
        int RandB(void){
        	int B = (rand() % (MAX_RAND + 1 - MIN_RAND)) + MIN_RAND;
            return B;
        }
        \end{lstlisting}
        En ajoutant une boncle FOR dans le main, on peut ainsi test beaucoup plus de valeurs.
        \begin{lstlisting}[style=CStyle]
        	for(int i = 0 ; i < 200000 ; i++){
        		A = RandA();
        		B = RandB();
        		printf("%d\t%d\t%d\t%d\n", i, A, B,PGCD(A, B));
        	}
        \end{lstlisting}
        
        Avec l'ajout de cette fonctionnalité, j'ai pu remarqué que la fonction PGCD ne prenait pas en compte les cas où $A = 0$ ou $B = 0$.
        Conformément à l'annexe du sujet, la fonction PGCD devient alors :
        \begin{lstlisting}[style=CStyle]
        int PGCD(int A, int B)
        {
        	while(A != B){
        		if (A==0) return B; 
        		if (B==0) return A;
        		if (A > B){
        			A = A - B;
        		} else {
        			B = B - A;
        		}
        	}
        	return A;
        }
        \end{lstlisting}
    
    \subsection{Etape 3}
    
    \subsection{}
    
\section{Hardware}
    \subsection{Etape 1}
    
\section{Conclusion} % Conclusion


\newpage %-------Saut de Page---------

\printbibliography % References

\end{document}
